\section{Superconduttivit\`a}
Per introdurre la teoria della superconduttivit\`a non si pu\`o fare a meno di fare una prima panoramica sulle evidenze fenomenologiche che si sono susseguite nel corso degli anni. Le evidenze fenomenologiche hanno portato alla formulazione delle prime \textit{leggi fenomenologiche} in grado di spiegare, in modo accurato, alcune evidenze. La prima teoria fenomenologica di \textbf{London} prevedeva in modo molto buono l'espulsione di flusso di campo magnetico ma, come vedremo, aveva altri punti molto deboli. I punti deboli della teoria di London, vengono rafforzati dalla teoria fenomenologica di \textit{Ginzburg-Landau}, ammettendo quindi la possibilit\`a dell'esistenza di materiali superconduttori, in cui vi era penetrazione di campo magnetico. Su questa linea, tracciata da Ginzburg e Landau, si posiziona anche Abrikosov, con appunto l'introduzione dei famossissimi \textit{Vortici di Abrikosov}. Il tutto sfocier\`a nella realizzazione di una teria microscopica da parte di Bardeen-Cooper-Schrieffer (BCS), in cui \`e evidente che, nonostante le teorie fenomenologiche evessero dalla loro la previsione abbastanza accurata e precisa, dei fenomeni, avevano il punto di partenza completamente errato. La teoria microscopica delle coppie di Cooper apre una nuova interpretazione del fenomeno della superconduttivit\`a. 
\subsection{Evidenze Fenomenologiche}
Le evidenze fenomenologiche della superconduttivit\`a possono essere riassunte in sei punti.
\subsubsection{Evidenza della superconduttivit\`a nel Mercurio raffreddato a 4.2K}
Per semplicit\`a identifichiamo con \textbf{N} un meteriale Normale, con \textbf{S} un materiale superconduttore. Per i materiali \textbf{N}, la curva di resistivit\`a in funzione alla temperatura \`e rappresentata in modo molto buono dalla legge sperimentale
\newl{\rho(T) = \rho_o + B T^5.}
Questa dipendenza, \`e facilmente ricavabile, per esempio, facendo esperimenti di resistivit\`a sull'argento. I materiali \textbf{S} sono caratterizzati dall'esistenza di un temperatura $T_c$ chiamata \textit{temepratura critica}, tale che per $T<T_c$, la funzione di resistivit\`a scritta prima ha una discontinuit\`a e crolla a zero. Questo \`e stato il caso del Mercurio raffreddato ad una temperatura inferiore a $4.2K$. Fino a quella temperatura mostra un comportamento tipico con $\rho(T)\sim T^5$ ma da $T<T_c$, si ha che $\rho(T)=0$. Questo fatto, fu osservato per la prima volta nel 1911, dai fisici (??).

\subsubsection{Effetto Maissner-Ochsenfeld}
L'effetto Meissner-Ochsenfeld \`e un altro fatto fenomenologico, legato al fenomeno superconduttivo. Un materiale \textbf{N}, immerso in un campo magnetico, nel momento di transizione di fase per $T<T_C$ espelle quasi completamente tutto il flusso di campo magnetico che lo attraversa. Per maggior chiarezza, avviene quanto rappresentato in Fig.~\ref{MEIS:FIG}. Questo fenomeno \`e alla base della lievitazione magnetica.
\begin{figure}
	\centering
	\fbox{
	\includegraphics[width=120mm,angle=0,clip=,scale=0.5]{IMG/Meissner.png}
	}
	\caption{Rappresentazione di quello che avviene alle linee di campo magnetico per $T<T_C$}
	\label{MEIS:FIG}
\end{figure}

\subsubsection{Calore specifico}
Dal punto di vista della comprensione microscopica della superconduttivit\`a, fu particolarmente importante osservare la curva di calore specifico di un materiale superconduttore.
\begin{figure}
	\centering
	\fbox{
		\begin{tikzpicture}[scale=1,auto=center]
			\draw[->] (-2,0) -- (2,0);
			\draw[->] (-2,0) -- (-2,5);
			\node[] at (2,-0.25) {$T$};
			\node[] at (-2.5,5) {$C_v(T)$};
			\draw[domain=-2:0] plot (\x,{1.91^(\x+2)-1});
			\draw[domain=0:2] plot (\x,{0.7*\x+(2*0.7)});
			\draw[<->,dashed] (0,1.91^2-1) -- (0,-0.5); 
			\node[] at (0,-0.7) {$T_C$};
		\end{tikzpicture}
	}
	\caption{Calore specifico per un superconduttore}
	\label{SUPER:CAL:SP}
\end{figure}
Come \`e possibile osservare in Fig.~\ref{SUPER:CAL:SP}, per $T<T_C$ il calore spercifico ha un'andamento esponenziale, mentre per $T>T_C$ di tipo lineare. Questo si presta ad una interpretazione davvero affascinante in quanto \`e un diretto richiamo al modello di \textit{Cristallo di Einstein}, in pi\`u il fatto che in $T=T_C$ sia presente una discontinuit\`a, fa pensare che \textit{da qualche parte}, sia presente un gap nello spettro di energie. A chiarire questi dubbi sar\`a direttamente la teoria microscopica di BCS. La parte esponenziale per temperature inferiori alla temperatura critica, si vedr\`a essere diretta causa del fatto che la superconduttivit\`a \`e strettamente legata al verificarsi di una interazione attrattiva tra di elettroni, vicnini alla superficie di fermi, mediata da un fonone.

\subsubsection{Principio isotopico}
Il principio isotopico \`e una ulteriore prova sperimentale che il fenomeno della presenza di portatori di carica, che generano la \textit{superconrrente}, \`e strettamente legato alle interazioni col reticolo. Sperimentalmente si osserva che la temperatura critica e la massa ionica sono correlate tra loro da una legge di tipo
\newl{T_C = \frac{1}{\sqrt{M^{a^+}}}.}
Come mostrato dall'andamento del calore specifico, anche questo fatto, indica che alla base del fenomeno ci deve essere una stretta interazione col reticolo cristallino, in particolar modo, come si vedr\`a, una interazione tra gli elettroni, mediata dal fonone.

\subsubsection{Esistenza di due tipi di superconduttori}
Altra importante evidenza fenomenologica, consiste nell'esistenza di due tipi differenti di superconduttori e che il fenomeno della superconduttivit\`a \`e legato anche ad un campo mangetico critico, da cui dipende $T_C$.
Come si pu\`o notare in Fig.~\ref{SUPER:TIPI2} sono rappresentati i comportamenti dei due tipi differenti di superconduttori. Nei superconduttori di tipo II, quando si \`e tra la regione compresa tra i due campi magnetici critici, si ha uno stato misto, in cui il materiale \`e si un superconduttore ma si ha penetrazione di campo magnetico. In questa situzione di presentano i noti vortici di Abrikosov. 
\begin{figure}[H]
	\centering
	\begin{subfigure}[b]{0.4\textwidth}
	\fbox{
		\begin{tikzpicture}[scale=1,auto=center]
			\draw[->] (-2,0) -- (2,0);
			\draw[->] (-2,0) -- (-2,5);
			\node[] at (2,-0.25) {$T$};
			\node[] at (-2.75,5) {$H$};
			\draw[domain=-2:0] plot (\x,{-0.5*(\x+2)^2 +2});
			\draw[<->,dashed] (0,0.5) -- (0,-0.5); 
			\node[] at (0,-0.7) {$T_C$};
			\node[] at (-2.5,2) {$H_C$};
		\end{tikzpicture}
	}
	\caption{TIPO I - Andamento di $H_C$ in funtione a $T$. Diamagnete perfetto, completo effetto Maissner}
	\end{subfigure}
	\qquad\quad
	\begin{subfigure}[b]{0.4\textwidth}
	\fbox{
		\begin{tikzpicture}[scale=1,auto=center]
			\draw[->] (-2,0) -- (2,0);
			\draw[->] (-2,0) -- (-2,5);
			\node[] at (2,-0.25) {$T$};
			\node[] at (-2.75,5) {$H$};
			\draw[domain=-2:0] plot (\x,{-0.5*(\x+2)^2 +2});
			\draw[domain=-2:0] plot (\x,{-(\x+2)^2 +4});
			\draw[<->,dashed] (0,0.5) -- (0,-0.5); 
			\node[] at (0,-0.7) {$T_C$};
			\node[] at (-2.5,2) {$H_{C1}$};
			\node[] at (-2.5,4) {$H_{C2}$};
		\end{tikzpicture}
	}
	\caption{TIPO II - Andamento di $H_C$ in funtione a $T$. Creazione di uno stato misto. vortici di Abrikosov}
	\end{subfigure}
	\qquad\quad
        \begin{subfigure}[b]{0.4\textwidth}
	\fbox{
		\begin{tikzpicture}[scale=1,auto=center]
			\draw[->] (-2,0) -- (2,0);
			\draw[->] (-2,0) -- (-2,5);
			\node[] at (2,-0.25) {$H$};
			\node[] at (-2.75,5) {$-4\pi M$};
			\draw[domain=-2:0] plot (\x,{\x+2});
			\draw[<->,dashed] (0,1.91^2-1) -- (0,-0.5); 
			\node[] at (0,-0.7) {$H_C$};
		\end{tikzpicture}
	}
	\caption{TIPO I - Andamento della Magnetizzazione n funzione di $H$}
	\end{subfigure}
	\qquad\quad
        \begin{subfigure}[b]{0.4\textwidth}
	\fbox{
		\begin{tikzpicture}[scale=1,auto=center]
			\draw[->] (-2,0) -- (2,0);
			\draw[->] (-2,0) -- (-2,5);
			\node[] at (2,-0.25) {$H$};
			\node[] at (-2.75,5) {$-4\pi M$};
			\draw[domain=-2:0] plot (\x,{\x+2});
			\draw[domain=0:1] plot (\x,{2/(\x+1)});
			\draw[<->,dashed] (0,1.91^2-1) -- (0,-0.5); 
			\draw[<->,dashed] (1,1.91^2-1) -- (1,-0.5);
			\node[] at (0,-0.7) {$H_{C1}$};
			\node[] at (1,-0.7) {$H_{C2}$};
		\end{tikzpicture}
	}
	\caption{TIPO II - Andamento della Magnetizzazione n funzione di $H$}
	\end{subfigure}
	\caption{Comportamenti differenti dei due tipi di superconduttori}
	\label{SUPER:TIPI2}
\end{figure}

\subsubsection{Esistenza delle correnti persistenti - Intrappolamento di flusso di campo magnetico}
Una ulteriore prova dell'esistenza del fenomeno della superconduttivit\`a sono le \textit{correnti persistenti}. Supponiamo di prendere un anello di materiale superconduttore in equilibrio, immergiamolo in un campo magnetico, per effetto Meissner, si verr\`a a creare una corrente in modo tare che si venga a creare un campo magnetico che annulli quello interno. In questo regime, supponiamo di spegnere il campo magnetico esterno, quello che si verifica \`e che la corrente, non trovando modo di dissiparsi, continua a viaggiare nel materiale superconduttivo intrappolando tra la sua area, una cerca quantit\`a di campo magnetico. Si misura sperimentalmente che la quantit\`a di flusso concatenato, dovuto alle correnti persistenti \`e \textbf{quantizzato} e proporzionale a
\newl{\Phi=\frac{hc}{2e}.}
Questo risultato \`e particolarmente importante e sar\`a necessario per effettuare dei ragionamenti sulle teorie fenomenologiche che verranno esposte nel prossimo paragrafo.
\subsection{Teoria Fenomenologica di London}
La teoria fenomenologica di London \`e la prima proposta di spiegazione di spiegazione del fenomeno della superconduttivit\`a. Come sar\`a possibile vedere, presenta numerosi limiti e non spiega tutte le evidenze fenomenologiche ma deriva in modo molto preciso l'effetto Maissner per i superconduttori di tipo-I e il fatto che il flusso di campo magnetico intrappolato \`e quantizzato. Di notevole importanza sar\`a osservare la usa forma, molto simile a quella sperimentale, ma differente in un dettaglio sostanziale. London, inizia considerando che nel materiale superconduttore in studio, ci sia una certa porzione di elettroni che rappresentano i portatori di carica della supercorrente. Per semplicit\`a si identifichi questa densit\`a di super-elettroni con $n_s$, quindi la densit\`a di supercorrente \`e $J=-n_s e v$.  Si consideri che la super-corrente identificata da questi elettroni sia un fluido incomprimibile, quindi $\nabla J = 0$. Per descrivera la dinamica dei super-elettroni, si usi la legge di Lorentz
\newl{\frac{dv}{dt} = -\frac{e}{m}\left[E+\frac{1}{c}v\times B \right], }
che sviluppando la derivata totale di $v$ diventa:
\newl{\frac{dv}{dt}  + \nabla\left(\frac{1}{2}v^2\right) - v\times(\nabla \times v) = -\frac{e}{m}\left[E+\frac{1}{c}v\times B \right]. }
Riordinando semplicemente i termini
\newl{\frac{dv}{dt} + \nabla\left(\frac{1}{2} v^2\right) = -\frac{e}{m} E + v\times\left[ \nabla \times v -\frac{e}{mc}B  \right]   
	\label{LOR:SUP}
}
Si definisce Bulk della Superconduttivit\`a  la quantit\`a $Q= \nabla \times v -e/(mc)B$. Usando la seconda eq di Maxwell $\nabla \times E = -(1/c) (\partial_t B)$ \`e possibile scrivere
\newl{\frac{\partial Q}{\partial t} = \nabla \times (v\times Q).
	\label{EQUIL:SUP}
}
\`E possibile ora fare alcuni ragionamenti sulla quantit\`a $Q$. In assenza di campo magnetico, col materiale superconduttore in equilibrio, \`e sostanzialmente tutto fermo e la quantit\`a $Q=0$. Applicando un campo magnetico si ha che per effetto Maissner il superconduttore raggiunge un suo equilibrio, l'Eq.~(\ref{EQUIL:SUP}) ci indica che l'equilibrio raggiunto \`e totalmente trasparente al modo in cui lo si \`e raggiunto, quindi $Q$ rimane sempre nulla per ogni superconduttore all'equilibrio. In questo modo abbiamo le \textit{equazioni di London}, la prima deriva dal fatto che per ogni superconduttore si ha che $Q=0$,
\newl{\textbf{I Eq. London }\boxed{\nabla \times v - \frac{e}{mc}B=0.} \label{PRIM:EQ:L} }
La seconda equazione di London si determina semplicemente inserendo l'Eq.~(\ref{PRIM:EQ:L}) nel'Eq.~(\ref{LOR:SUP}) ottenendo
\newl{\textbf{II Eq. London }\boxed{\frac{dv}{dt} + \nabla\left(\frac{1}{2} v^2\right) = -\frac{e}{m}E}}
\subsubsection{Risultati della Teoria di London}
Dalla prima equazione di London si deriva subito un risultato molto importante sull'effetto Maissner. Prendiamo l'Eq.(\ref{PRIM:EQ:L}), scrivendola in funzione di $J$ compaiono una densit\`a di super-elettroni e una carica elettrica. Il rotore di $J$ \`e un laplaciano con un fattore $4\pi$. Detto questo, la prima equazione di London pu\`o essere riscritta come
\newl{\nabla^2 B(x) = \frac{mc}{4\pi n_s e^2 }B(x),}
La cui soluzione \`e particolarmente semplice
\newl{B(x) =B_0 e^{-\frac{x}{\lambda_L} },}
dove $\lambda_L$ \`e la lunghezza di penetrazione di London che nell'ordine di $\lambda_L\sim50nm$.

Sempre la prima equazione di London fornisce un'informazione molto importante sul momento dei super-elettroni. Sostituendo semplicemente, $\nabla \times A = B$ nella prima equazione di London, si ottiene che:
\newl{\nabla \times \left(p-\frac{e}{c}A\right)= \nabla \times P = 0, \label{ROT:MOM:NUL}}
che \`e un risultato che servir\`a tra poco.
Si consideri ora la seconda equazione di London linearizzata, quindi consideriamo di eliminare il termine in $\nabla v^2$
\newl{\frac{dv}{dt} = -\frac{e}{m} E, }
usando la seconda equazione di Maxwell e considerando la derivata rispetto alla densit\`a di corrente, anzich\`e le sole velocit\`a la scrittura si arricchisce divendando
\newl{\nabla\times\frac{dJ}{dt} = \frac{n_se^2}{mc}\frac{\partial B}{\partial t}.}
Integrando sulla superficie attraversata dal campo magnetico si ottiene
\newl{\int_{\Sigma} \left(\nabla\times\frac{dJ}{dt}\right) d\Sigma - \frac{n_se^2}{mc}\int_{\Sigma} \left(\frac{\partial B}{\partial t}\right) d\Sigma =0   
\\
\oint_{\gamma} \frac{dJ}{dt} dl - \frac{n_se^2}{mc} \int_{\Sigma} \left(\frac{\partial B}{\partial t}\right) d\Sigma =0
\\
\frac{d}{dt}\left( \int_{\Sigma} B \cdot d\Sigma -\frac{mc}{n_se^2}\oint_{\gamma} J \cdot dl \right)=0.
}
Si definisce \textit{FLUSSOIDE}, e lo si indica con la lettera $\Phi$, la quantit\`a
\newl{\Phi=\int_{\Sigma} B \cdot d\Sigma -\frac{mc}{n_se^2}\oint_{\gamma} J \cdot dl. \label{FLUSSOIDE}}
Ricordando l'Eq.~(\ref{ROT:MOM:NUL}) e usandola nell'Eq.~(\ref{FLUSSOIDE}) otteniamo uno dei risultati pi\`u importanti della teoria di London, cio\`e che il flusso di campo mangetico, in un superconduttore, \`e quantizzato. Facendo il conto si ottiene
\newl{\Phi_L=\int_{\Sigma} B \cdot d\Sigma -\frac{mc}{n_se^2}\oint_{\gamma} J \cdot dl = \int_\Sigma \left(\nabla\times A\right)\cdot d\Sigma - \frac{mc}{e}\oint_{\gamma} v\cdot dl = 
\\
=\oint_{\gamma} A \cdot dl - \frac{mc}{e}\oint_{\gamma} v\cdot dl = \oint_{\gamma}\left(A - \frac{mc}{e} v\right)\cdot dl =
\\
=-\frac{c}{e}\oint_{\gamma} \left(mv - \frac{e}{c} A\right) dl = -\frac{c}{e}\int_\Sigma\left( \nabla \times P\right) d\Sigma.
}
Dato che $\nabla\times P = 0$, l'ultimo passaggio ci indica che $\Phi$ \`e una costante. Applicando il teorema di Stokes, si ottiene esattamente l'equazione di Bohr-Sommerfeld per l'azione ridotta quantistica
\newl{\Phi_L=-\frac{e}{c}\int_\Sigma\left( \nabla \times P\right) d\Sigma= -\frac{c}{e} \oint_{\gamma}P\cdot dl = \frac{c}{e}2\pi \hbar n}
L'equazione di Bohr-Sommerfeld ha come autovalori multipli di $\hbar$, quindi risulter\`a che il flusso sar\`a multiplo di
\newl{\boxed{\Phi_L = \frac{hc}{e}.}}
\subsubsection{Conclusioni sulla teoria di London}
Come \`e stato possibile vedere, la teoria di London prevede bene l'esistenza dei superconduttori di tipo-I soggetti ad effetto Maissner. Prevede che la quantizzazione del flusso di campo magnetico intrappolato nel superconduttore, ma \`e sbagliato di un fattore due. Infatti, ricordando i risultati fenomenologici ottenuti con la misura del campo magnetico intrappolato in una anello superconduttore si ha che 
\newl{\boxed{\Phi_{sperimentale} = 2\Phi_L.}}
In questa relazione \`e riassunto il grande limite della teoria. Non spiega chi sono i portatori di carica. Non sono elettroni, sono un qualcosa con carica doppia dell'elettrone. Nonostante siano gi\`a in questo modo evidenti alcuni aspetti della superconduttivit\`a, il fatto che non siano previsti superconduttori di tipo-II e il fatto che non si riesca a valutare la natura dei portatori di carica, suggerisce il fatto che l'approcio classico al problema \`e sbagliato. Nel prossimo paragrafo si vedr\`a come questa teoria fenomenologica, pu\`o essere corretta, e raffinata inserendo le correzioni di Ginzburg-Landau all'energia libera. Nonostante gli sforzi, rimane concettualmente sbagliato il punto di partenza semi-classico di queste teoria.
\subsection{Le equazioni di Pippard}
Come visto nel precedente paragrafo, la teoria fenomenologica di London ha dei limiti abbastanza grossolani. Il risultato sul flussoide, scalato di un fattore due, dal punto di vista di una teoria fenomenologica, era un problema facilmente risolvibile attraverso l'introduzione di grandezze efficaci. Il limite pi\`u grande della teoria riguarda appunto la mancata previsione dei superconduttori di tipo-II. Si introdue il cos\`i detto gauge di London
\newl{
\nabla \cdot A = 0\nonumber\\
A\cdot n = 0.
\label{GAUGE:LOND}
}
In questo gauge, la prima equazione di London diventa
\newl{
	\int_{\Sigma}\left( \nabla \times J\right)\, d\Sigma = -\frac{n_s e^2}{mc}\int_{\Sigma}\left(\nabla \times A\right)\, d\Sigma \nonumber\\
	\int_{\gamma} J \cdot dl = -\frac{n_s e^2}{mc} \int_{\gamma} A \cdot dl\nonumber\\
	J(x) = - \frac{n_s e^2}{mc} A(x).
	\label{SUPERCORRENTE}
}
Si ha quindi una realzione tra la densit\`a di corrente e il potenziale vettore nello stesso punto. Gli studi di Pippard lo portarono a concludere che la lunghezza di penetrazione del campo mangetico nel superconduttore non \`e una costante. Secondo la teoria di London, cambiare la penetrazione di campo magnetico voleva dire cambiare il numero di$ n_s$, quindi stravolgere in modo sensibile le propriet\`a termodinache dei superconduttori. Pippard propone una lettura della supercorrente in Eq.~\ref{SUPERCORRENTE} in modo locale, dove la $J(x)$ \`e in funzione ad un campo medio di $A(x)$ primo vicino di dimensione $r_0$. Per elementi adeguatamente dopati, $r_0$ \`e paragonabile al cammino medio di un elettrone libero $l$, nel caso dei metalli Pippard definisce una \textit{lunghezza di correlazione} identificata con $\xi_0$. La relazione di Pippard lega insieme $r_0$, $l$ e $\xi_0$
\newl{\frac{1}{r_0}=\frac{1}{l}+\frac{1}{\xi_0} }
la densit\`a di supercorrente \`e possibile considerarla come
\newl{J(x) = -\frac{n_S e^2}{mc} \frac{3}{4\pi \xi_0} \int d^3x\, \frac{X[X\cdot A(x')]t}{\abs{X} ^4} e^{-x/r_0} }
[continuare inserendo gli estremi per ottenere effetto Meissner...]




\subsection{Teoria fenomenologica di Ginzburg-Landau}
Come \`e stato possibile osservare nel paragrafo precedente, la teoria fenomenologica di London ha alcuni limiti come la mancata previsione dell'esistenza dei superconduttori di tipo-II e l'errata stima del flussoide di un fattore 2. Lo stesso London aveva notato questo limite ad esempio nelle giunzioni (sia tra due superconduttori che tra un superconduttore e un materiale normale). Quando lo strato di separazione tra i due layer della giunzione diventava molto piccolo sostanzialmente era impossibile da spiegare il fatto che la superficie di energia diventava positiva. London cerc\`o di valutarne l'entit\`a tramite degli studi fatti su giunzioni Sn-In, anche se una iniziale trattazione, sempre fenomenologica, fu fatta da Ginzburg-Landau.
\subsubsection{Espanzione dell'energia libera}
L'energia libera di un superconduttore $F_s(T,H)$ dipender\`a dalla temperatura e dal campo magnetico applicato. Il superconduttore \`e un perfetto diamagnete, quindi il contributo dato da $H$ sar\`a puramente diamagnetico, permettendo quindi di scrivere
\newl{F_S(T,H) \sim F_S(T,0)+\frac{H^2}{8\pi}.}
Si consideri ora l'energia libera di un materiale normale e la si identifichi con $F_N(T,H)$. Anche lei dipender\`a dalla temperatura e dal campo magnetico. Per $T > T_C$ il contributo magnetico \`e molto piccolo, quindi trascurabile permettendo di scrivere
\newl{F_N(T,H)\sim F_N(T,0).}
Per $T\sim T_C$ si ha che $F_N \sim F_S$ quindi
\newl{
	F_S(T,H) \sim F_N(T,0) + \frac{H^2}{8\pi}. 
	\label{EN:LIB}	
}
La proposta di Ginzburg-Landau \`e quella di descrivere l'energia libera di un superconduttore in funzione ad un parametro complesso d'ordine che definiamo essere $\Psi(x)$. Il modulo di $\abs{\Psi} $ ha la propriet\`a di essere piccolo per $T\sim T_C$ quindi l'energia libera in Eq.~\ref{EN:LIB}, in assenza di campo magnetico, \`e possibile descriverla come sviluppo in funzione del parametro d'ordine
\newl{
	F_S(T,0) = F_N(T,0) + a \abs{\Psi} ^2 + \frac{b}{2} \abs{\Psi} ^4 +\cdots,
	\label{SVILUPPO:GL}
}
in cui $a$ e $b$ sono due costanti fenomenologiche dipendenti dalla temperatura. Ginzburg e Landau, analogamente all'equazione di Schoedinger, aggiungono allo sviluppo un termine di ordine $\abs{\nabla \psi} ^2$ che ha appunto la forma
\newl{(2m^*)^{-1}\abs{\left(-i\hbar\nabla+\frac{e^*}{c}A\right) \Psi } ^2.  }
Riutilizzando tutti i risultati nell'Eq.~\ref{SVILUPPO:GL}, si ottiene
\newl{F_S = F_{N,0} + a \abs{\Psi} ^2 + \frac{b}{2} \abs{\Psi} ^4 + (2m^*)^{-1}\abs{\left(-i\hbar\nabla+\frac{e^*}{c}A\right) \Psi } ^2 + \frac{H^2}{8\pi}}
in cui, tutte le grandezze asteriscate sono delle grandezze efficaci definite come
\newl{
	m^* = 2m_e \nonumber\\
	e^* = 2e   \nonumber\\
	\abs{\Psi} = n_s^* = \frac{1}{2}n_s \nonumber
}
dove $n_s$ rappresenta la densit\`a di elettroni coinvolti nel fenomeno della superconduzione.
[continuare riprendendo la densit\`a di corrente quantistica di Pippard...]

\subsection{Teoria BCS }
[Inserire discussione qualitativa sulle coppie di Cooper, funzione d'onda BCS e interazione attrattiva tra gli elettroni, mediata dal fonone, vista come perturbazione elastica in seconda quantizzazione...]
\subsubsection{Calcolo del gap di energia}
Come \`e stato gi\`a possibile notare In Fig.~\ref{SUPER:CAL:SP}, il calore specifico presenta una discontinuit\`a in $T_C$ questo implica che lo spettro delle energie dei super-portatori di carica presenta un gap di energie proibite. \`E possibile determinare lo spettro delle energie risolvendo il semplice problema agli autovalori
\newl{
	\left[-\frac{\hbar^2}{2m} \left(\nabla_{x1}^2 + \nabla_{x2}^2\right) + V(r_2-r_1)  \right]\Psi(r_2-r_1) = E\Psi(r_2-r_1)
	\label{HAM:BCS}
}
dove le funzioni d'onda che si stanno considerando sono di tipo BCS
\newl{
	\Psi(r_2-r_1)=\int_{\Omega} \frac{d^3k}{(2\pi)^3} \chi(k) e^{ik(r_2-r_1)},
	\label{WAW:BCS}
}
il potenziale che si sta considerando \`e di tipo attrattivo e periodico. Per semplicit\`a ci si riferisce al centro di massa della coppia di Cooper quindi $k_1=-k_2$. Sostituendo la funzione d'onda in Eq.~\ref{WAW:BCS} Eq.~\ref{HAM:BCS} \`e possibile identificare alcune forme note. Concentrandosi solo sulla parte di potenziale si ottiene proprio
\newl{
	V(r_2-r_1)\int_{\Omega} \frac{d^3k}{(2\pi)^3} \chi(k) e^{ik(r_2-r_1)} = \int_{\Omega} \frac{d^3k}{(2\pi)^3} \chi(k)V(r_2-r_1) e^{ik(r_2-r_1)}=\nonumber\\
	=\left(\int_{\Omega} \frac{d^3k}{(2\pi)^3} \chi(k)V(r_2-r_1) e^{i(k-k')(r_2-r_1)}\right)e^{ik'(r_2-r_1)}=\nonumber\\
	=\int_{\Omega}\frac{d^3k'}{(2\pi)^3}\left(\int_{\Omega} \frac{d^3k}{(2\pi)^3} \chi(k)V(r_2-r_1) e^{i(k-k')(r_2-r_1)}\right)e^{ik'(r_2-r_1)}=\nonumber\\
	=\int_{\Omega}\frac{d^3k'}{(2\pi)^3}\chi(k') V(k,k')e^{ik'(r_2-r_1),}
}
dove $V(k,k')$ \`e  la trasformata di Fourier di $V(r_2-r_1)$. L'integrazione che \`e stata fatta su $k'$ non fa altro che ambientare tutta l'Hamiltoniana in Eq.~\ref{HAM:BCS} nello spazio $k$. Gli elettroni che sentono l'interazione attrattiva sono quelli pi\`u esterni alla sfera di Fermi. L'energia che possono scambiare con un fonone \`e nell'ordine di $\hbar\omega_D = 10^{-2} eV$ mentre l'energia di Fermi \`e nell'ordine $1\div5\,eV$, quindi solo gli elettroni che si trovano in una corona circolare spessa $\hbar\omega_D$ dalla sfera di fermi, possono risentire del potenziale attrattivo, gli altri elettroni interni non hanno modo di interagire coi fononi. Seguendo questo ragionamento, senza perdere di generalit\`a \`e possibile descrivere il potenziale $V(k,k')$ come
\newl{
	V(k,k') &=& -U\;\;\;  \forall k \in [k_F,k_F+k_D]\nonumber\\
	V(k,k') &=& 0 \;\;\;\;\;  \text{ Altrimenti}.
}
Come detto in precedenza, l'Hamiltoniana nello spazio $k$, sostituendo banalmente la variabile di integrazione in $k' \to k$, \`e possibile scriverla come
\newl{ 
	\frac{\hbar^2}{2m} k'^2\chi(k') + \int_{\Omega}\frac{d^3k}{(2\pi)^3}\chi(k) V(k,k')e^{ik(r_2-r_1)}=E\chi(k')\nonumber\\
	\left[\frac{\hbar^2}{2m}k'^2 -E\right]\chi(k') = U\int_{k_F}^{k_F+\Delta k}\frac{d^3k}{(2\pi)^3}\chi(k) e^{ik(r_2-r_1)}\nonumber
}
Integrando in $k'$ su tutto lo spazio, sostituendo ancora $k'\to k$ e passando in cordinate sferiche si ottiene
\newl{
	1 = \frac{4\pi U}{(2\pi)^3}\int_{k_F}^{k_F+\Delta k}d^3k\frac{k^2}{\frac{\hbar^2}{2m}k^2 -E}.
}
Passando alle energie tramite la sostituzione $k\to\sqrt{\xi}(\sqrt{2k}/\hbar)$ , $\xi \to (\hbar^2/2m) k^2$ e quindi $d\xi = (\hbar^2/m) k dk$. Chiamando $\Delta = 2E_F - E$ si ottiene
\newl{
	1=\frac{U}{2\pi^2}\int_{E_F}^{E_F+\hbar\omega_D} \left(\frac{2m}{\hbar^2}\right)^{\frac{3}{2}} \sqrt{\xi} (2\xi-E)^{-1}\,d\xi \sim\nonumber\\
	\sim\frac{U}{2\pi^2} g_{3D}(E_F)\int_{E_F}^{E_F+\hbar\omega_D} (2\xi -E)^{-1}\,d\xi =\nonumber\\ 
	=\frac{U}{2\pi^2}g_{3D}(E_F)\ln{\abs{\frac{\Delta + 2\hbar\omega_D}{\Delta}} }
}
da cui si ricava direttamente un'espressione del gap di energie
\newl{\boxed{\Delta(U) = \frac{2\hbar\omega_D}{\exp{\left[\frac{2}{Ug_{3D}(E_F)}\right] }-1}. }}












