
\section{ Esperimento Stern-Gerlach }
Il problema consiste nella deflessione di un fascio di particelle neutre immerse in un campo magnetico disomogeneo; 
quello che ci si aspetterebbe \`e una distribuzione omogenea di particelle sullo schermo. Tuttavia quello che si 
osserva \`e che il numero di componenti in cui viene diviso il fascio pu\`o esser pari o dispari. 
\\
\\
Supponiamo un campo magnetico $\vec{B}$ diretto come $z$, e $\dpart{B}{z} \neq 0$. 
\\
La forza che agisce sull'atomo in presenza di campo magnetico disomogeneo \`e: 
$$ F_z = \mu_z \dpart{B}{z} $$
Pensiamo all'atomo come una spira percorsa da corrente (l'elettrone che si muove su una circonferenza attorno al nucleo),
in questo caso il mio atomo \`e dotato di un campo magnetico che posso esprimere come:
$$ \vec{\mu} = \vec{I} A \hat{n} $$
dove $A=\pi r^2$ \`e l'area dell'orbita, $\vec{I}=\frac{q}{T}=\frac{q\vec{v}}{2\pi r}$ la corrente generata dal moto
della particella di carica $q$ e $\hat{n}$ la normale uscente dal piano della spira.
\\
Dato che $\vec{l}=mr\vec{v}\hat{n}$ posso riscrivere il momento magnetico in modo classico come:
$$ \vec{\mu} = \frac{qr\vec{v}}{2}\hat{n} = \frac{q}{2m}\vec{l} $$
Di fatto questa proporzionalit\`a tra $\vec{\mu}$ e $\vec{l}$ vale anche nel caso di operatori quantistici;
per un elettrone di carica $q=-\abs{q_e}$ posso scrivere:
$$ \opvec{\mu_l} = \frac{- \abs{q_e} }{2 m_e} \opvec{l} $$
\\
Introduco la seguente costante, nota come \emph{magnetone di bohr}:
$$ \mu_B \equivdef \frac{\hbar q_e}{2m_e} $$
Allora il momento orbitale di un elettrone diventa:
$$ \opvec{\mu_l} = -\mu_B\frac{\opvec{l} }{\hbar} $$
\\

